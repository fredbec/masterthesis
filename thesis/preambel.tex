\documentclass[11pt]{article}
\usepackage{thumbpdf,lmodern}
\usepackage[utf8]{inputenc}
%\usepackage[margin=3.3cm]{geometry}
%\renewcommand{\familydefault}{\sfdefault}
%\usepackage[top=2.5cm, left=2.8cm, right=2.8cm, bottom=2.5cm]{geometry}
\usepackage[margin=3cm]{geometry}

%%% PACKAGES
\usepackage[dvipsnames]{xcolor}
\usepackage{graphicx} % support the \includegraphics command and options
\usepackage{booktabs} % for much better looking tables
\usepackage{array} % for better arrays (eg matrices) in maths
\usepackage{paralist} % very flexible & customisable lists (eg. enumerate/itemize, etc.)
\usepackage{verbatim} % adds environment for commenting out blocks of text & for better verbatim
\usepackage{tikz}
\usepackage[all]{xy}
\usepackage{mathtools}
\usepackage{url}
\def\UrlBreaks{\do\/\do-} % this is for lineabreaking of bibliography urls
\usepackage{hyperref}
\usepackage{color}
\usepackage{float}
\usepackage[justification=centering]{caption}
\usepackage{subcaption}
\usepackage{booktabs}
\usepackage{acro}
\usepackage{setspace}
\usepackage{eurosym}
\usepackage{listings} %% for R Code
\usepackage{natbib}
%\usepackage{csquotes}
%\usepackage{footnote}
\usepackage{amsmath}
\usepackage{amsfonts}
\usepackage{amsthm}
\usepackage{courier}
\usepackage{appendix}
\usepackage{listings}


%%% Listings options
\lstdefinelanguage{Renhanced}[]{R}{
  otherkeywords={!,!=,~,\$,*,\&,\%/\%,\%*\%,\%\%,<-,<<-, ::},
  morekeywords={},
  deletekeywords={hist, runif, plot, read.table, read, check, text, file, attributes, quote, missing, c, list, any, which, na, deparse, structure, install, model, data, sub, family},
  alsoletter={.\%},%
  alsoother={:_\$}}

 \lstset{
  language=Renhanced,                     % the language of the code
  basicstyle=\small\ttfamily, % the size of the fonts that are used for the code
  numbers=left,                   % where to put the line-numbers
  numberstyle=\tiny\color{Blue},  % the style that is used for the line-numbers
  stepnumber=1,                   % the step between two line-numbers. If it is 1, each line will be numbered
  numbersep=10pt,                  % how far the line-numbers are from the code
  backgroundcolor=\color{white},  % choose the background color. You must add \usepackage{color}
  showspaces=false,               % show spaces adding particular underscores
  showstringspaces=false,         % underline spaces within strings
  showtabs=false,                 % show tabs within strings adding particular underscores
  frame=false,                   % adds a frame around the code
  rulecolor=\color{black},        % if not set, the frame-color may be changed on line-breaks within not-black text (e.g. commens (green here))
  tabsize=2,                      % sets default tabsize to 2 spaces
  captionpos=b,                   % sets the caption-position to bottom
  breaklines=true,                % sets automatic line breaking
  breakatwhitespace=false,        % sets if automatic breaks should only happen at whitespace
  keywordstyle=\color{RoyalBlue},      % keyword style
  commentstyle=\color{YellowGreen},   % comment style
  stringstyle=\color{ForestGreen}      % string literal style
}



%%% BibTex Style
\setcitestyle{authoryear, open = { ( }, close = { ) }}
\def\bibfont{\small} % smaller bibliography

% Equation numbering
\numberwithin{equation}{section}

% French spacing
\frenchspacing



\renewcommand{\lstlistingname}{Code-Chunk}

%%% New commands
\newcommand{\li}{\lstinline}
\newcommand{\estf}{\hat{\boldsymbol{f}}}
\newcommand{\estbbeta}{\hat{\boldsymbol{\beta}}}
\newcommand{\bbeta}{\boldsymbol{\beta}}
\newcommand{\by}{\mathbf{y}}
\newcommand{\bx}{\mathbf{X}}
\newcommand{\btau}{\boldsymbol{\tau}}
\newcommand{\balpha}{\boldsymbol{\alpha}}
\newcommand{\xib}{\mathbf{x}_{i}' \bbeta}
\newcommand{\btheta}{\boldsymbol{\theta}}
\newcommand{\boldeta}{\boldsymbol{\eta}}
\newcommand{\XB}{\mathbf{X} \boldsymbol{\beta}}
\newcommand{\bgamma}{\boldsymbol{\gamma}}



%%% HEADERS & FOOTERS
\usepackage{fancyhdr} % This should be set AFTER setting up the page geometry
\pagestyle{fancy} % options: empty , plain , fancy
\renewcommand{\headrulewidth}{0pt} % customise the layout...
\lhead{}\chead{}\rhead{}
\lfoot{}\cfoot{\thepage}\rfoot{}



%%% SECTION TITLE APPEARANCE
% (This matches ConTeXt defaults)

%%% ToC (table of contents) APPEARANCE
\usepackage[nottoc,notlof,notlot]{tocbibind} % Put the bibliography in the ToC
\usepackage[titles]{tocloft} % Alter the style of the Table of Contents
\renewcommand{\cftsecfont}{\rmfamily\mdseries\upshape}
\renewcommand{\cftsecpagefont}{\rmfamily\mdseries\upshape} % No bold!

\usepackage{setspace}
\onehalfspacing

%for todos
\usepackage{xcolor}
\newcommand\todos[1]{\textcolor{red}{#1}}

%\renewcommand{\arraystretch}{0.85}
\title{Master Thesis}
\author{Friederike Becker}